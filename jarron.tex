\titlename{Jarrón}

Uno no se lamenta por lo que en el instante se rompe, uno se entristece
por las cosas que fueron o pudieron ser.

Heredas un jarrón y éste se suicida, desde una repisa, entusiasmado por
la cola de un gato.

Ves el adorno estallado en más de mil partes que todavía bailan en el piso,
como monedas recién caídas. Llorás, llorás desconsoladamente. Es el que 
sostuvo las flores que le regalaron a tu mamá el día que te parió. El 
jarrón que le dio agua a las margaritas que te regaló el chico que andaba
en bicicleta cuando tenías once, las mismas que faltaban, como piezas
perdidas, en el jardín rompecabezas de su vecina. El que contuvo al primer
ramo de rosas que te regaló tu novio, cuando cumpliste veintiún años, las 
que se marchitaron, como todo lo sano.

Y también se partió el alma desde el estante, porque iba a ser el mismo
jarrón, que mudarías al centro de la mesa de tu nueva casa.

La culpa no es del gato, ni de la elección del lugar. Son jarrones hechos
para lo frágil de algún tiempo. Como respirar.
