\titlename{Recién Nacida}

A las 9 de la noche fue a la terminal con un bolso en el que no entraban
más de tres mudas. Esquivó palomas y compró un pasaje: sí, con 
descuento universitario, asiento individual, chau.

Le gusta viajar de noche, piensa que el tiempo pasa más rápido. En la
butaca llena de ácaros malgastó la batería de su celular que marcaba
el último porciento y se durmió buscando una comodidad que no encontró.

A la madrugada se despertó por la ruta destruída; señal de que habían
hecho mitad de trayecto. Miró por la ventanilla cómo la luna seguía su
viaje y escuchó las chicharras enloquecidas que anunciaban la llegada
de diciembre. ``Este cielo no es el que se puede ver allá'', pensó. A los
quince minutos, que quizás fueron dos horas, el pueblo empezó a asomar:
las vías, el olor a eucalipto y los reductores de velocidad sincronizaron
en un bostezo.

Bajó en la terminal de paredes descascaradas, y entre manchas de aceite
caminó hasta un taxi. El chofer le ofreció ayuda; lo reconoció, fueron
juntos a la primaria pero pasados los doce o trece años no lo volvió a
ver. Él le sonrió pero ella se limitó a indicar coordenadas. Cuando la
bajada de bandera señaló once pesos comenzó a buscar sus llaves. 

Calle a calle sintió una presión en el pecho que aumentó en cada esquina.
Cierre, botón, abrojo y otra vez cierre. Unos pocos autos circulaban.
Respiró profundo y revisó de nuevo sus bolsillos. Las llaves no aparecieron
ni entonces ni después. Se sentía apurada pese a no tener ninguna obligación
u horario que cumplir. Al llegar, le pagó al taxista y cerró la puerta sin
saludar.

Los perros le dieron la bienvenida con ladridos y saltos. La casa le
pareció más grande que la de su memoria. Dejó de buscar las llaves cuando
recordó que las había perdido en la mudanza. Se puso a llorar como una
recién nacida, lloró kilometros, lloró todo el destierro y también el
futuro. Del lado de adentro se escuchó la cerradura destrabarse. Una voz
le dijo: ``Ya estás acá''.
