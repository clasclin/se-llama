\titlename{Guerra tácita}

En el piso del edificio éramos tres: un chico músico, un abuelo y yo.
Todo transcurria en movimiento rutinarios. Los miércoles la empresa de agua,
los jueves el correo, una vez cada treinta días las expensas, los viernes y
sábados el músico con su banda, los domingos gritos de gol por las ventanas
y ahí concluía el bullicio. El viejo y yo, los silenciosos, fuimos comandantes
y soldados de una guerra tácita.

Después de las siete de la tarde cada de cada día se sacaba la basura; debía
depositarse en un cuartito frente a las escaleras y una hora más tarde el 
encargado pasaría a buscarla.

Cada vez que sacaba mi bolsa de desechos encontraba la puerta de ese minúsculo
espacio abierta. No puedo explicar si era el olor, lo estrictamente desprolijo
o qué clase de trastorno obsesivo compulsivo lo que me obligaba a cerrar la
puerta, respirar profundo y soltar el picaporte. Todas las tardes lo mismo.

Los lunes el músico dejaba algunas botellas de vino o gaseosa y durante la 
semana se ausentaba. Muchas veces el abuelo esperaba que se sacara más bolsas
y cerrara la puerta para salir él, minutos después, y dejarla abierta. Yo lo
espiaba por las mirilla y me fastidiaba con una facilidad que hoy me da gracia.
Él sabía que lo observaba.

Pasaron los meses y el desacuerdo era constante. Yo ya había pegado un
papelito ``por favor, mantener cerrado''. Se puso amarillo sin conseguir mi 
objetivo.

Una tarde, después de dejar mis residuos embolsados, me fui de casa y no
regresé a dormir.

Al día siguiente, la puerta del cubículo seguía cerrada y en el espejo del
ascensor una nota firmada por la administración: ``Sentimos e informamos la
pérdida de nuestro querido vecino, durante quince años, José 'Pepe' Robles
del 5to 'B'''.

Estuve en silencio varios minutos antes de entrar a mi casa. ``Gané'' ---dije
en voz alta--- en el mismo instante que la bisagra empezó a chillar y la 
puerta del basurero se abrió.
