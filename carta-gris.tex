\titlename{Carta Gris}

Era un viernes de esos en los que uno se encuentra obligado a parecer una
persona. Me levanté apurado sin desayuno ni ritual de amor. No pude sostenerle 
la mirada ni verla descansar porque la noche anterior, también, había dormido 
solo. Cuando bajaba por el asensor me acordé que la gente se peina, se perfuma
y cuando quise abrir los ojos ya estaba sentado en mi escritorio. El panorama
de siempre y la sonrisa de los miserables como yo. Cuando el sol cayó nos
devolvieron la libertad hasta el siguiente lunes. Volví a mi casa arrugado y
arrastrado. Antes de recaer en la cama sentí como infiltraban un sobre gris por
debajo de la puerta. Me levante sin ánimos, pensando que era alguna volanta de
descuentos o un dos por uno en una parrilla de dueños chinos, pero no, era una
carta. Ahí estaba yo, con la correspondencia en mis manos inestables, 
transpiradas. Tímido pero seguro. Sin culpas de meterme para siempre en lo que
no fue. Moví la cortaplumas removiendo el lacre con una violencia cuidadosa,
semejante a la ansiedad de quien está cambiando su vida. Ella nunca existió.
Lo que mi mente recuerda\ldots\newline

\begin{quotation}
\noindent
A quien corresponda:\\

Hola, ¿qué hacés despierto? No es la primera vez que hago esto,
pero la primera que sé, más o menos, lo que puedo encontrar. Lo que quiero. Yo
a vos no te conozco, vos a mi tampoco. Pero te vi, te leí, te seguí. Te escribo
desde la salida de emergencia del vigésimo primer piso de donde vivís. Me veo
conversando con vos una tarde, sobre lo que te pasó durante el día. 
Distrayéndote de la cama, para que no estés semi dormido. Pensando cómo perder
el tiempo hasta que sea lunes de vuelta, para que no te quejes de tener que
irte de nuevo, no tener tiempo, de no vivir. Si estás expectante, deseando que
esta carta se convierta en un té con miel o te estás imaginando cómo hacer para
verme, te espero, como nunca, en la esquina de mala muerte donde venden sushi y
algo más. El viernes que viene a esta hora. No me digas que no.
\firma{María.}
\end{quotation}

Los días de esa semana fueron un cuentagotas de minutos eternos. Me llené los
espacios de imaginación pensando cómo había estado en la puerta de mi departamento 
y en vez de aparecer había dejado una carta. \emph{una carta} invitandome a 
conocernos.

Cuando, ese viernes, llegué a 9 y 45 no vi nada. La esquina misteriosa donde
vendían sushi estaba convertida en baldío. Estaba seguro que esa era la esquina.
La casualidad o concidencia hicieron que ella también la llame de esa forma ``la
esquina de mala muerte donde\ldots'' Un baldío que me rompió los ojos en el mismo
momento que desplegué la carta de mi bolsillo y vi que los espacios de las
coordenadas estaban borroso, como si la tinta suicida hubiera saltado. Como si
nadie lo hubiese escrito.

